\documentclass[openany,12pt,a4paper]{book}
\usepackage{graphicx}
\usepackage{amsmath, amssymb, amsthm}
\usepackage{amsfonts}
\usepackage{extarrows}
\usepackage{booktabs}
\usepackage{makecell}
\usepackage{geometry}
\usepackage{float}
\usepackage{multirow,multicol}
\geometry{a4paper,left=2cm,right=2cm}

\begin{document}
\chapter{Generic Product Development Process}
\begin{enumerate}
    \item[0.] \textbf{Planning:} Identify the need for a new product or improvement of an existing product
    \item \textbf{Concept Development:} Generate and evaluate ideas for new products or improvements
    \item \textbf{System-Level Design:} Define the architecture and components of the product
    \item \textbf{Detail Design:} Create detailed specifications for each component and subsystem
    \item \textbf{Testing and Refinement:} Test the product and make necessary adjustments
    \item \textbf{Production Ramp-Up:} Prepare for full-scale production and launch the product
\end{enumerate}
% Topic 4
\chapter{Opportunity Identification}
\begin{itemize}
    \item The 5-steps of in Planning Process:
    \begin{itemize}
        \item \textbf{Identify opportunities}
        \item Evaluate and prioritize projects
        \item Allocate resources and plan timing
        \item Complete pre-project planning
        \item Reflect on the results and the process
    \end{itemize}
\end{itemize}
\section{Opportunity}
\begin{itemize}
    \item Definition: An idea for a new product, a hypothesis about how value might be created
    \item Ansoff's Growth Matrix:
    \begin{table}[H]
        \centering
        \begin{tabular}{c|c|c}
            \toprule
            & \textbf{Current Products} & \textbf{New Products} \\
            \midrule
            \makecell[c]{\textbf{Current}\\\textbf{Markets}} & \makecell[c]{Market\\Penetration\\strategy} & \makecell[c]{Product\\Development\\strategy} \\
            \hline
            \makecell[c]{\textbf{New}\\\textbf{Markets}} & \makecell[c]{Market\\Development\\strategy} & \makecell[c]{Diversification\\strategy} \\
            \bottomrule
        \end{tabular}
    \end{table}
    \begin{itemize}
        \item Market Penetration Strategy: Focus on increasing sales of existing products in existing markets
        \item Product Development Strategy: Focus on developing new products for existing markets
        \item Market Development Strategy: Focus on entering new markets with existing products
        \item Diversification Strategy: Focus on developing new products for new markets
        \item Risk: Current $<$ New, Product development $>$ Market penetration 
    \end{itemize}
    \item Opportunity tournaments: A structured process for generating and evaluating new product ideas
    \begin{itemize}
        \item Advantages:
        \begin{itemize}
            \item Generate a large number of ideas
            \item Seek high quality of opportunities generated
            \item Create high variance in the quality of opportunities
        \end{itemize}
    \end{itemize}
\end{itemize}

\section{Opportunity Identification}
\begin{itemize}
    \item Ulrich and Eppinger's 6-step process:
    \begin{enumerate}
        \item Establish a charter
        \item Generate and sense many opportunities
        \item Screen opportunities
        \item Develop promising opportunities
        \item Select exceptional opportunities
        \item Reflect on the results and process
    \end{enumerate}
\end{itemize}

\subsubsection{Step 1: Establish a Charter}
\begin{itemize}
    \item Charter: Articulate the goals and establish the boundary conditions for an innovation effort
    \item Charter $\approx$ Mission statement for a new product
    \item Requires:
    \begin{itemize}
        \item Resolving a tension between leaving the innovation problem unconstrained
        \item Specifying a direction that is likely to meet the goals of the team and organization
    \end{itemize}
    \item Recommended:
    \begin{itemize}
        \item The innovation charter be broad. Benefit is that opportunities that may otherwise have never been considered will challenge the team’s assumptions about what kinds of opportunities it should pursue
    \end{itemize}
\end{itemize}
\subsubsection{Step 2: Generate and Sense Many Opportunities}
\begin{itemize}
    \item Opportunities from various sources:
    \begin{itemize}
        \item Internal sources: Employees, R\&D, existing products
        \item External sources: Customers, competitors, market trends, technology advancements
    \end{itemize}
    \item Techniques for Generating Opportunities:
    \begin{itemize}
        \item Follow a Personal Passion
        \item Compile Bug Lists
        \item Pull Opportunities from Capabilities
        \item Study Customers (Find latent needs)
        \item Consider Implications of Trends
        \item Imitate, but Better
        \item Mine Your Sources (Mainly external sources)
        \begin{itemize}
            \item Lead users
            \item Representation in social networks
            \item Universities and government laboratories
            \item Online idea submission
        \end{itemize}
    \end{itemize}
\end{itemize}

\subsubsection{Step 3: Screen Opportunities}
\begin{itemize}
    \item Goal: To eliminate opportunities that are highly unlikely to result in the creation of value and to focus attention on the opportunities worthy of further investigation
    \item \textbf{Not} to pick the single best opportunity
    \item Two methods for screening opportunities:
    \begin{itemize}
        \item Web-based surveys
        \item Workshops with ``multivoting''
    \end{itemize}
\end{itemize}

\subsubsection{Step 4: Develop Promising Opportunities}
\begin{itemize}
    \item Goal: To resolve the greatest uncertainty surrounding each one at the lowest cost in time and money
    \item Determine:
    \begin{itemize}
        \item The major uncertainties regarding the success of each opportunity
        \item The tasks you could take to resolve the uncertainties
        \item The approximate cost of each task
    \end{itemize}
    \item Invest modest levels of resources in developing a few of them
    \item Additional tasks (customer interviews, testing of existing products, etc.)
\end{itemize}

\subsubsection{Step 5: Select Exceptional Opportunities}
\begin{itemize}
    \item Method: RWW (Real, Win, Worth it)
    \begin{itemize}
        \item Real: Is there a real market and a real product?
        \item Win: Can we win? Can our product or service be competitive? Can we succeed as a company?
        \item Worth it: Is it worth doing? Is the return adequate and the risk acceptable?
    \end{itemize}
\end{itemize}

\subsubsection{Step 6: Reflect on the Results and the Process}
\begin{itemize}
    \item How many of the opportunities identified came from internal sources versus external sources?
    \item Did we consider dozens or hundreds of opportunities?
    \item Was the innovation charter too narrowly focused?
    \item Were our filtering criteria biased, or largely based on the best possible estimates of eventual product success?
    \item Are the resulting opportunities exciting to the team?
\end{itemize}

\section{R\&D}
\begin{itemize}
    \item R\&D: Reasearch and Development
    \item Main activities
    \begin{enumerate}
        \item Discovering and developing new technologies
        \item Improving understanding of the technology in existing products
        \item Improving and strengthening understanding of technologies used in manufacturing
        \item Understanding research results from universities and other research institutions
    \end{enumerate}
\end{itemize}

\chapter{Product Planning}
\begin{itemize}
    \item An activity that considers both the current product line and the potential portfolio of projects that an organization might pursue
\end{itemize}
\section{Product Planning Process}
\begin{itemize}
    \item Product plan: Identifies the portfolio of products to be developed by the organization and the timing of their introduction to the market
    \item Inefficiencies (no good product plan):
    \begin{itemize}
        \item Inadequate coverage of target markets with competitive products
        \item Poor timing of market introductions of products
        \item Mismatches between aggregate development capacity and the number of projects pursued
        \item Poor distribution of resources, with some projects overstaffed and others understaffed
        \item Initiation and subsequent cancellation of ill-conceived projects
        \item Frequent changes in the directions of projects
    \end{itemize}
\end{itemize}
\subsection{Types of Product Plans}
\begin{itemize}
    \item New Product Platforms: A set of products that share a common architecture and components, allowing for economies of scale and scope
    \item Derivatives of existing product platforms: Products that are based on existing platforms but have modifications or enhancements
    \item Incremental improvements to existing products: Small enhancements or modifications to existing products to improve performance, quality, or features
    \item Fundamentally new products: Products that are significantly different from existing offerings and may require new technologies or processes
\end{itemize}
\section{Process}
\begin{enumerate}
    \item Identify Opportunities
    \item Evaluate and Prioritize Projects
    \item Allocate Resources and Plan Timing
    \item Complete Pre-Project Planning
    \item Reflect on the Results and the Process
\end{enumerate}
% Page 73
\subsection{Step 1: Identify Opportunities}
\begin{itemize}
    \item Opportunity funnel
    \item Recommend: each promising opportunity be described in a short, coherent statement and that this information be collected in a database
\end{itemize}

\subsection{Step 2: Evaluate and Prioritize Projects}
\begin{itemize}
    \item To select the most promising projects to pursue
    \item Basic perspectives:
    \begin{itemize}
        \item Competitive strategy
        \item Market segmentation 
        \item Technological trajectories
        \item Project platform
    \end{itemize}
\end{itemize}
\subsubsection{Competitive Strategy}
\begin{itemize}
    \item Defines a basic approach to markets and products with respect to competitors
    \begin{itemize}
        \item Technology leadership
        \item Cost leadership
        \item Customer focus
        \item Imitate
    \end{itemize}
\end{itemize}
\subsubsection{Market Segmentation}
\begin{itemize}
    \item Allows the firm to consider the actions of competitors and the strength of the firm’s existing products with respect to each well-defined group of customers
    \item Can assess which product opportunities best address weaknesses in its own product line and which exploit weaknesses in the offerings of competitors
\end{itemize}
\subsubsection{Technological Trajectories}
\begin{itemize}
    \item Key: When to adopt a new basic technology in a product line
    \item Conceptual tool: Technology S-Curve
\end{itemize}
\subsubsection{Project Platform Planning}
\begin{itemize}
    \item The set of assets shared across a set of products
    \item Effective platform $\xrightarrow[Rapidly]{Easily}$ A variety of derivative products
    \item Key Strategy: Whether any project will develop a derivative product from an existing platform or develop an entirely new platform
    \item Technology roadmap: 
\end{itemize}

\subsubsection{Evaluating Fundamentally New Product Opportunities}
\begin{itemize}
    \item Market size (units/year$\times$average price)
    \item Market growth rate (percent per year)
    \item Competitive intensity (range of competitors and their strengths)
    \item Depth of the firm's existing knowledge of the market
    \item Depth of the firm's existing knowledge of the technology
    \item Fit with the firm's other products
    \item Fit with the firm's core assets and capabilities
    \item Potential for patents, trade secrets, or other barriers to competition
    \item Existence of a product champion within the firm
\end{itemize}

\subsubsection{Balancing the Portfolio}
\begin{itemize}
    \item Two specific dimensions:
    \begin{itemize}
        \item The extent to which the project involves a change in the product line
        \item The extent to which the project involves a change in production processes
    \end{itemize}
    \item Advantage:
    \begin{itemize}
        \item Be useful to illuminate imbalances in the portfolio of projects under consideration 
        \item In assessing the consistency between a portfolio of projects and the competitive strategy
    \end{itemize}
\end{itemize}

\subsection{Step 3: Allocate Resources and Plan Timing}
\subsubsection{Allocate Resources}
\begin{itemize}
    \item Aggregate Planning: Helps an organization make efficient use of its resources by pursuing only those projects that can reasonably be completed with the allocated resources
    \item Primary resource to be managed: The effort of the development staff (person-hours or person-months)
    \item Capacity utilization: 80\% to 90\%
\end{itemize}
\subsubsection{Project Timing}
\begin{itemize}
    \item Timing of product introductions
    \item Technology readiness
    \item Market readiness
    \item Competition
\end{itemize}
\subsubsection{The Product Plan}
\begin{itemize}
    \item The set of projects approved by the planning process, sequenced in time
    \item May include:
    \begin{itemize}
        \item A mix of fundamentally new products
        \item Platform projects
        \item Derivative projects of varying size
    \end{itemize}
\end{itemize}
\subsection{Step 4: Complete Pre-Project Planning}
\begin{itemize}
    \item By: Core team
    \item Product vision statement: A brief description of the product and its intended market
\end{itemize}
\subsubsection{Mission Statement}
\begin{itemize}
    \item Brief (one sentence) description of the product
    \item Benefit proposition
    \item Key business goals
    \item Target market(s) for the product
    \item Assumptions and constraints that guide the development effort
    \item Stakeholders
\end{itemize}
\subsubsection{Assumptions and Constraints}
\begin{itemize}
    \item Considers the strategies of several functional areas within the firm
    \item Consider:
    \begin{itemize}
        \item Manufacturing
        \item Service
        \item Environment
    \end{itemize}
\end{itemize}

\subsection{Step 5: Reflect on the Results and the Process}
\begin{itemize}
    \item Is the opportunity funnel collecting an exciting and diverse set of product opportunities?
    \item Does the product plan support the competitive strategy of the firm?
    \item Does the product plan address the most important current opportunities facing the firm?
    \item Are the total resources allocated to product development sufficient to pursue the firm's competitive strategy?
    \item Have creative ways of leveraging finite resources been considered, such as the use of product platforms, joint ventures, and partnerships with suppliers?
    \item $\cdots$
\end{itemize}

\chapter{Concept Development}
\begin{enumerate}
    \item \textbf{Identify Customer Needs}
    \item \textbf{Establish Target Specifications}
    \item \textbf{Generate Product Concepts}
    \item \textbf{Select Product Concepts}
    \item \textbf{Test Product Concepts}
    \item \textbf{Set Final Specifications}
    \item \textbf{Plan Downstream Development}
\end{enumerate}
\section{Specifications}
\begin{itemize}
    \item Customer needs $\to$ ``Language of the customer''
    \item Specifications $\to$ ``Language of the engineer''
    \item Specifications: an unambiguous agreement on what the team will attempt to achieve to satisfy the customer needs
    \item Include: Metric, Value
\end{itemize}
\section{Establishing Target Specifications}
\begin{itemize}
    \item Target specifications: The goals of the development team, describing a product that the team believes would succeed in the marketplace
    \item 4 steps:
    \begin{itemize}
        \item Prepare the list of metrics
        \item Collect competitive benchmarking information
        \item Set ideal and marginally acceptable target values
        \item Reflect on the results and the process
    \end{itemize}
\end{itemize}
\section{Concept Generation}
\subsection{The Activity of Concept Generation}
\begin{itemize}
    \item Concept generation: An approximate description of the technology, working principles, and form of the product
    \item Form (with a brief textual description):
    \begin{itemize}
        \item Sketch
        \item rough three-dimensional model
    \end{itemize}
\end{itemize}
\subsection{Structured Approaches}
\begin{itemize}
    \item 5 steps:
    \begin{enumerate}
        \item Clarify the problem
        \item Search externally
        \item Search internally
        \item Explore systematically
        \item Reflect on the results and the process
    \end{enumerate}
\end{itemize}
\section{Concept Selection}
\begin{itemize}
    \item Concept selection: The process of evaluating concepts with respect to customer needs and other criteria, comparing the relative strengths and weaknesses of the concepts, and selecting one or more concepts for further investigation, testing, or development
    \item 5 stages:
    \begin{enumerate}
        \item Initial Screen
        \begin{itemize}
            \item Technical
            \item Research direction and balance
            \item Competitive rationale
            \item Patentability
            \item Stability of the market
            \item Integration and synergy
            \item Market
            \item Channel fit
            \item Manufacturing
            \item Financial
            \item Strategic fit
        \end{itemize}
        \item Customer Screen
        \item Technical Screen
        \item Final Screen
        \item Business Analysis
    \end{enumerate}
\end{itemize}

\section{Concept Testing}
\begin{itemize}
    \item 7 steps:
    \begin{enumerate}
        \item Define the purpose of the concept test
        \item Choose a survey population
        \item Select a survey format
        \item Communicate the concept
        \item Measure customer response
        \item Interpret the results
        \item Reflect on the results and the process
    \end{enumerate}
\end{itemize}

\section{Setting Final Specifications}
\begin{itemize}
    \item Final specifications: Revised target specifications after product concept selection
    \item 5 steps:
    \begin{itemize}
        \item Develop technical models of the product
        \item Develop a cost model of the product
        \item Refine the specifications, making trade-offs where necessary
        \item Flow down the specifications as appropriate
        \item Reflect on the results and the process
    \end{itemize}
\end{itemize}

\section{Output -- The Contract book}
\begin{itemize}
    \item Mission statement
    \item Customer needs
    \item Details of the selected concept
    \item The product specifications
    \item The economic analysis of the product
    \item The development schedule
    \item The project staffing
    \item The budget
\end{itemize}

\section{Ongoing Activities}
\begin{itemize}
    \item Economic analysis
    \item Benchmarking
    \item Modeling
\end{itemize}

\chapter{System-Level Design}
\section{Product Architecture}
\begin{itemize}
    \item Production Architecture: The assignment of the functional elements of a product to the physical building blocks of the product
    \begin{itemize}
        \item Functional elements
        \item Physical elements (chunk)
    \end{itemize}
    \item Types:
    \begin{itemize}
        \item Modular Architecture
        \begin{itemize}
            \item Chunks implement one or a few functional elements in their entirety
            \item The interactions between chunks are well defined and are generally fundamental to the primary functions of the product
        \end{itemize}
        \item Integral Architecture
        \begin{itemize}
            \item Functional elements of the product are implemented using more than one chunk
            \item A single chunk implements many functional elements
            \item The interactions between chunks are ill defined and may be incidental to the primary functions of the products
        \end{itemize}
    \end{itemize}
\end{itemize}
\section{Modular Architecture}
\begin{itemize}
    \item Type:
    \begin{itemize}
        \item Slot-modular architecture
        \item Bus-modular architecture
        \item Sectional-modular architecture
    \end{itemize}
\end{itemize}
\section{Implications of Architecture}
\begin{itemize}
    \item Product Change
    \begin{itemize}
        \item Upgrade
        \item Add-ons
        \item Adaptation
        \item Wear
        \item Consumption
        \item Flexibility in use
        \item Reuse
    \end{itemize}
    \item Product Variety
    \item Component Standardization
    \item Product Performance
    \item Manufacturability
    \item Product Development Management
\end{itemize}
\section{Establish the Architecture}
\begin{enumerate}
    \item Create a schematic of the product
    \item Cluster the elements of the schematic
    \begin{itemize}
        \item Geometric integration and precision
        \item Function sharing
        \item Capabilities of vendors
        \item Similarity of design or production technology
        \item Localization of change
        \item Accommodating variety
        \item Enabling standardization
        \item Portability of the interfaces
    \end{itemize}
    \item Create a rough geometric layout
    \item Identify the fundamental and incidental interactions
\end{enumerate}

\chapter{Testing, ramp-up and Product Launch}
\section{Testing and Refinement}
\begin{itemize}
    \item construction and evaluation of multiple production versions of the product
\end{itemize}
\section{Production ramp-up}
\begin{itemize}
    \item Using the intended production system to make the product
    \item Purpose: To train the work force and to work out any remaining problems in the production processes
    \item Four Basic Types of Production Systems:
    \begin{itemize}
        \item Process layout - functional
        \item Product layout - line
        \item Cellular layout - group
        \item Fixed positions
    \end{itemize}
    \item Goal:
    \begin{itemize}
        \item Use space efficiently
        \item Efficient personnel movement
        \item Maximum equipment utilization
        \item Convenient / safe work environment
        \item Simplify repair / maintenance work
        \item Smooth flow of work
    \end{itemize}
\end{itemize}
\section{Product Launch}
\begin{itemize}
    \item 3W1H: When, Where, to Whom, How
    \item Minimum Viable Product (MVP): Enterprises develop product versions that are usable and can express core concepts with minimal cost
\end{itemize}
\section{Robust Design}
\begin{itemize}
    \item Robust product: Performs as intended even under nonideal conditions such as manufacturing process variations or a range of operating situations
    \item Robust design: the product development activity of creating a robust product
    \item Robust setpoint: A combination of design parameter values for which the product performance is as desired under a range of operating conditions and manufacturing variations
    \item Process:
    \begin{enumerate}
        \item Identify control factors, noise factors, and performance metrics
        \item Formulate an objective function
        \begin{itemize}
            \item Maximizing
            \item Minimizing
            \item Target value
            \item Signal-to-noise ratio
        \end{itemize}
        \item Develop the experimental plan
        \begin{itemize}
            \item Experimental plan
            \begin{itemize}
                \item Full factorial: $2^n$
                \item Fractional factorial
            \end{itemize}
        \end{itemize}
        \item Run the experiment
        \item Conduct the analysis
        \item Select and confirm factor setpoints
        \item Reflect and repeat
    \end{enumerate}
\end{itemize}

\chapter{Digital Transformation \& Digital Product}
\section{Digital Transformation}
\begin{itemize}
    \item Digitization: Process of converting information from analog to digital
    \item Digitalization: Process of using digitalized information to make established ways of working simpler and more efficient
    \item Digital Transformation: Process of using digital technology to create new or modify existing business processes, culture, and customer experience to meet changing business and market requirements
    \item Types:
    \begin{itemize}
        \item Process Transformation
        \item Business Model Transformation
        \item Domain Transformation
        \item Cultural/Organizational Transformation
    \end{itemize}
    \item Guildlines:
    \begin{itemize}
        \item Understand your technology
        \item Embrace Cultural Change
        \item Consider a new digital business model
        \item Digital upsklling
        \item Ensure collaboration
        \item Top management support
    \end{itemize}
    \item Advantage of digital product:
    \begin{itemize}
        \item Low investment, high return
        \item More profitable than physical goods
        \item No inventory, shipping or rent hassle
        \item Automated delivery for passive income
        \item Serve a niche at scale
        \item Digital products offer unique ways to communicate directly with the customers
    \end{itemize}
\end{itemize}
\section{Digital Project Development}
\begin{itemize}
    \item 6 steps:
    \begin{enumerate}
        \item Discovery
        \item Ideate
        \item Test
        \item Execute
        \item Launch
        \item Grow
    \end{enumerate}
\end{itemize}

\chapter{Managing R\&D}
\section{The Importance and Nature of R\&D}
\begin{itemize}
    \item Factors for the Evolution of R\&D Pattern
    \begin{itemize}
        \item Technology Explosion
        \item Shortening Technology Cycles
        \item Globalization of Technology
    \end{itemize}
\end{itemize}
\section{Strategic Role and Planning of R\&D}
\begin{itemize}
    \item Supporting existing businesses to maintain competitiveness
    \item Driving new business opportunities
    \item Conducting exploratory research to understand emerging technologies
    \item Factors to be consider when planning:
    \begin{itemize}
        \item Environmental forecasts (PEST)
        \item Comparative technological cost-effectiveness
        \item Risk assessment
        \item An analysis of the organization's capabilities
    \end{itemize}
\end{itemize}

\section{Classifying and Funding R\&D}
\begin{itemize}
    \item 2 main forms:
    \begin{itemize}
        \item Maintenance:
        \begin{itemize}
            \item Survival
            \item Competitiveness
        \end{itemize}
        \item Growth:
        \begin{itemize}
            \item Technology mastery
            \item Break the mould
        \end{itemize}
    \end{itemize}
    \item Key considerations for funding:
    \begin{itemize}
        \item Competitor spending
        \item Long-term growth objectives
        \item The need for stability
        \item Potential distortions from large projects
    \end{itemize}
    \item Common approaches
    \begin{itemize}
        \item Inter-firm comparisons
        \item A fixed relationship to turnover or profits
        \item An internal customer-contractor relationship
    \end{itemize}
\end{itemize}

\section{Evaluating R\&D Projects}
\begin{itemize}
    \item Challenge: The limitation of resources
    \item Use:
    \begin{itemize}
        \item Decision support software
        \item Expert judgment
    \end{itemize}
    \item Evaluation criteria:
    \begin{itemize}
        \item Technical feasibility
        \item Market stability
        \item Financial return
        \item Strategic fit
    \end{itemize}
\end{itemize}

\chapter{Patent \& IP}
\section{IP}
\begin{itemize}
    \item Intellectual Property (IP or IPR): An intangible asset created by human intellectual or inspirational activity
    \item Main types:
    \begin{itemize}
        \item Patents
        \item Trademarks
        \item Design
        \item Copyright
    \end{itemize}
\end{itemize}
\section{Patent}
\begin{itemize}
    \item 2 Types:
    \begin{itemize}
        \item Design patent
        \item Utility patent
        \begin{itemize}
            \item Useful
            \item Novel
            \item Nonobvious
        \end{itemize}
    \end{itemize}
    \item Cost:
    \begin{itemize}
        \item Annual fees
        \item Patent agents
        \item Court fees
    \end{itemize}
    \item Things can have patent:
    \begin{itemize}
        \item Process
        \item Machine
        \item Artical of manufacture
        \item Composition of matter
    \end{itemize}
    \item Benefits:
    \begin{itemize}
        \item Only the owner can benefit
        \item Owners can commercially exploit their ideas themselves or charge other organizations to use their patent
    \end{itemize}
    \item Patent harmonization:
    \begin{itemize}
        \item US: First to invent
        \item UK: First to file
    \end{itemize}
\end{itemize}
\section{Patent preparation process}
\begin{enumerate}
    \item Formulate a strategy \& plan
    \item Study prior inventions
    \item Outline claims
    \item Write the description
    \item Refine claims
    \item Pursue application
\end{enumerate}
\section{Registered Design}
\begin{itemize}
    \item Protect the outward appearance of an artical
\end{itemize}
\section{Trademark}
\begin{itemize}
    \item A distinctive name, mark or symbol that is identifiedwith a company's products
    \item Differentiating product
    \item Protect forever, bur need renew every 10 years
    \item Principle:
    \begin{itemize}
        \item Include: Distinctive color, name, symbol and trademark symbol
        \item Be distinctive
        \item Not be deceptive
        \item Not cause confusion
    \end{itemize}
    \item Trademark can give IP protection long after patent has lapsed
\end{itemize}
\section{Copyright}
\begin{itemize}
    \item 3 Categories:
    \begin{itemize}
        \item Original literary, dramatic, musical and artistic works
        \item Sound recordings, films, broadcasts and cable programmes
        \item The typographical arrangement or layout of a published edition
    \end{itemize}
    \item First creater or author or their employer owns it
    \item Cannot be copyright:
    \begin{itemize}
        \item Titles
        \item Facts
        \item Ideas
        \item Names
        \item Data
        \item Methods
        \item Systems
    \end{itemize}
\end{itemize}
\end{document}