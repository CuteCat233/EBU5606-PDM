\documentclass[openany,12pt,a4paper]{book}
\usepackage{graphicx}
\usepackage{amsmath, amssymb, amsthm}
\usepackage{amsfonts}
\usepackage{extarrows}
\usepackage{booktabs}
\usepackage{makecell}
\usepackage{geometry}
\usepackage{float}
\usepackage{multirow,multicol}
\geometry{a4paper,left=2cm,right=2cm}

\begin{document}
% Topic 1
\chapter{Introduction}
\section{Product}
\begin{itemize}
    \item Product Definition: 
    \begin{itemize}
        \item Service
        \item Item offered for sale
    \end{itemize}
    \item Form:
    \begin{itemize}
        \item Physical
        \item Virtual or Cyber
    \end{itemize}
    \item Provide:
    \begin{itemize}
        \item Problem-solving services
        \item Core benefits
    \end{itemize}
    \item Cost \& Price:
    \begin{itemize}
        \item Market
        \item Quality
        \item Marketing
        \item Segment that is target
    \end{itemize}
\end{itemize}

\section{Product Development}
\begin{itemize}
    \item Innovation: To renew, To make new, To alter
    \begin{itemize}
        \item Definition: A successful implementation of
        \begin{itemize}
            \item A new or significantly improved product or process
            \item A new marketing method or a new Organizational method
        \end{itemize}
        \item A creative process ($A + B + \cdots \xlongequal[way]{novel} \text{Unique new thing}$)
        \item Idea $\rightarrow$ Reality
        \item New inventions $\xrightarrow{Application}$ Marketable products and services
    \end{itemize}
    \item Six types of new product:
    \begin{itemize}
        \item New to world
        \item New product lines
        \item Additions to existing product lines
        \item Improvement and revisions to existing products
        \item Re-positionings
        \item Cost reductions
    \end{itemize}
    \item New Product Development: A subprocess of innovation
    \begin{itemize}
        \item Process: Bussiness Opportunities $\rightarrow$ Tangible product
    \end{itemize}
\end{itemize}

\section{Product Failure}
\begin{itemize}
    \item Important: Some ideas are not commercially viable or organizationally appropriate
    \item Reasons:
    \begin{itemize}
        \item Lack of innovation
        \item Insufficient budget
        \item Market misjudgment/demand misunderstanding
        \item Lack of management support
        \item Lack of customer participation
        \item Government policy intervention
        \item Misjudgment of market size
        \item Mismatch of company capabilities
        \item Insufficient channel support
        \item Listing delay
        \item Intense competition and countermeasures
        \item Inefficient organizational communication
        \item Insufficient investment return
        \item Sudden change in consumption trend
    \end{itemize}
    \item Avoid: Process improvements and a structured approach
    \begin{itemize}
        \item Better requirements capture and management
        \item Better planning
        \item Better analysis and screening
        \item Organization-wide process framework
        \item Better execution
    \end{itemize}
    \item Characteristics of Successful Product Development
\end{itemize}
    

\begin{table}[H]
    \centering
    \begin{tabular}{l|l}
        \toprule
        \textbf{Characteristic} & \textbf{Description}\\
        \midrule
        Product quality & Affects: Market share \& Price\\\hline
        Product Cost & \makecell[l]{Includes:\\1. Spending on capital equipment and tooling\\2. Incremental cost\\Determines:\\Profit for a sales volume and price}\\\hline
        Development time &\makecell[l]{Determines:\\1. Reaction capability\\2. The speed of obtaining returns\\\textbf{Plan future development times/schedules}} \\\hline
        Development cost & \makecell[l]{A significant fraction of the investment \\required to achieve the profits\\\textbf{Plan future budget and resources}}\\\hline
        Development capability & \makecell[l]{Determine:\\Efficiency and economy of product development\\\textbf{Reduce cost and time}}\\
        \bottomrule
    \end{tabular}
\end{table}

\section{Product Development Process}
\begin{itemize}
    \item Definition: A sequence of steps or activities which an enterprise employs to conceive, design, and commercialize a product
    \item Advantage of having a generic, well-defined process:
    \begin{itemize}
        \item \textbf{Quality assurance}: Specify the wise points and checkpoints (milestone)
        \item \textbf{Coordination}: Defines the roles of each of the players on the development team
        \item \textbf{Planning}: Natural milestone in process anchors the schedule of the overall development project
        \item \textbf{Managment}: Identify possible problem areas by comparing the actual events to the established process
        \item \textbf{Improvement}: Helps to identify opportunities for improvement
    \end{itemize}
    \item Phase:
    \begin{enumerate}
        \item[0.] Planning
        \begin{itemize}
            \item Assessment of technology developments and market objectives
            \item Output: Project Mission Statement (specifies the target market for the product, business goals, key assumptions and constraints)
        \end{itemize}
        \item Concept Development
        \begin{itemize}
            \item Identify the needs of the target market
            \item Alternative product concepts are generated and evaluated
            \begin{itemize}
                \item Concept: A description of the form, function and features of a product
            \end{itemize}
            \item Chose one or more concepts for further development and testing
            \item Use evaluation and screening to aid in the selection
            \begin{itemize}
                \item A set of specifications, an analysis of competitive products and an economic justification for the project
            \end{itemize}
        \end{itemize}
        \item System-level Design
        \begin{itemize}
            \item Include: Definition of the product architecture and the decomposition of the product into subsystems and components
            \item Output:
            \begin{itemize}
                \item A geometric layout of the product
                \item A functional specification of each of the products subsystems
                \item A preliminary process flow diagram for the final assembly process
            \end{itemize}
        \end{itemize}
        \item Detail Design
        \begin{itemize}
            \item Include: Complete specification of the geometry, materials and tolerances of all the unique parts of the product and any information regarding parts to be purchased from suppliers
            \item Output:
            \begin{itemize}
                \item Control documentation: The drawings or computer files describing the specifications of each of the parts of the product and how it is to be assembled
            \end{itemize}
            \item Critical issues:
            \begin{itemize}
                \item Production cost
                \item Robust performance
            \end{itemize}
        \end{itemize}
        \item Testing and Refinement
        \begin{itemize}
            \item Include: The construction and evaluation of multiple preproduction versions of the product
            \item Participants: Lead customer \& Employee
            \item Output: Feedback used to make improvements and adjustments to the products
            \item \textit{Alpha test}
        \end{itemize}
        \item Product ramp-up
        \begin{itemize}
            \item Purpose: To train the work force and to work out any remaining problems in the production process
            \item Participants: Preferred customers
            \item \textit{Beta test}
        \end{itemize}
    \end{enumerate}
    \item Decision point:
    \begin{itemize}
        \item Goal: Reduced cost and Prevent inferior products from entering the market
    \end{itemize}
    \item Key Departments:
    \begin{itemize}
        \item Marketing: Mediates the interactions between the firm and its customer
        \item Design (R\&D): Defining the physical form of the product and how this can best meet customer needs
        \item Manufacturing: Design and operation of the production of the product
    \end{itemize}
\end{itemize}

\section{Product Development Team}
\begin{itemize}
    \item Members: Representatives from each of these areas
    \item Made up:
    \begin{itemize}
        \item Core team: A team leader and one representative from each of the areas involved in all stages
        \item Extended team: all of the people involved in the development
    \end{itemize}
\end{itemize}

% Topic 2
\chapter{Product and Service Strategies}
\section{Product}
\begin{itemize}
    \item Product Definition: Anything that can be offered to a market for attention, acquisition, use or consumption that might satisfy a want or need
    \item Includes: Physical objects, Services, Persons, Places, Organizations and Ideas
    \item Levels of Product:
    \begin{enumerate}
        \item Core Product: The fundamental benefit or service that the customer gains from the product
        \item Actual Product: The tangible aspects of the product, including design, features, packaging, and branding
        \item Augmented Product: Additional services or benefits that enhance the product's value, such as warranty, customer support, and after-sales service
        \item[E.] Potential Product: Future enhancements or innovations that could be added to the product to meet changing customer needs or market trends
    \end{enumerate}
    \item Product Classification:
    \begin{itemize}
        \item Consumer Products: Products purchased by final consumers for personal consumption
        \begin{itemize}
            \item Convenience Products: Low-priced, frequently purchased items with minimal effort (e.g., groceries)
            \item Shopping Products: Higher-priced items that require comparison shopping (e.g., clothing, electronics)
            \item Specifialty Products: Unique items with specific characteristics that consumers actively seek (e.g., luxury cars, designer clothing)
            \item Unsought Products: Products that consumers do not think about regularly or may not know about (e.g., life insurance, funeral services)
        \end{itemize}
        \item Industrial Products: Products purchased for further processing or for use in conducting a business
        \begin{itemize}
            \item Materials and Parts: Raw materials and components used in manufacturing (e.g., steel, electronic components)
            \item Capital Items: Long-lasting goods that facilitate the development or production of other products (e.g., machinery, buildings)
            \item Supplies and Services: Operating supplies and services that support the production process (e.g., maintenance supplies, consulting services)
        \end{itemize}
        \item Other Marketable Entities: Products that do not fit neatly into the above categories but are still marketed
    \end{itemize}
\end{itemize}

\section{Product Decisions}
\begin{itemize}
    \item Individual Product Decisions:
    \begin{itemize}
        \item Product Attributes: Features, quality, design, and packaging that define the product
        \item Branding: The name, logo, and image associated with the product
        \item Packaging: The design and materials used to contain and protect the product
        \item Labeling: Information provided on the product packaging, including instructions, ingredients, and branding
        \item Product Support Services: Additional services that enhance the product's value, such as warranties, customer support, and after-sales service
    \end{itemize}
    \item Product Attribute Decisions:
    \begin{itemize}
        \item Quality: The overall excellence or superiority of the product, including performance, durability, and reliability
        \item Features: Specific characteristics or functionalities that enhance the product's appeal and usability
        \item Design: The aesthetic and functional aspects of the product, including its appearance, usability, and ergonomics
    \end{itemize}
    \item Branding Equity:
    \begin{itemize}
        \item Association: The connections and perceptions that consumers have with the brand
        \item Loyalty: The degree to which consumers consistently choose the brand over competitors
        \item Credibility: The trustworthiness and reliability of the brand in the eyes of consumers
        \item Awareness: The extent to which consumers recognize and recall the brand
    \end{itemize}
    \item Advantage of Brand name:
    \begin{itemize}
        \item Attributes: The specific features and characteristics associated with the brand
        \item Consistency: The reliability and predictability of the brand's performance and quality
        \item Quality \& Value: The perceived worth and benefits that the brand provides to consumers
        \item Identification: The ability of the brand to stand out and be recognized in the market
    \end{itemize}
    \item Major Branding Decisions:
    \begin{itemize}
        \item Brand name selection (selection and protection): Choosing a name that resonates with consumers and is legally protected
        \item Brand sponsor (manufacturer's brand, private brand, licensed brand, co-brand): Deciding whether to use a manufacturer's brand, a private label, a licensed brand, or a co-branding strategy
        \item Brand strategy: Determining how to expand or diversify the brand through line extensions, brand extensions, multibrands, or creating new brands
        \begin{itemize}
            \item Line extension: Existing brand name used for a new product in the existing category (e.g., a soft drink brand launching a new flavor)
            \item Brand extension: Using an existing brand name to enter a new product category (e.g., a clothing brand launching a fragrance)
            \item Multibrands: Offering multiple brands within the same product category to target different market segments (e.g., a company offering various detergent brands)
            \item New brands: Creating entirely new brands to enter new markets or product categories (e.g., a tech company launching a new line of smart home devices)
        \end{itemize}
    \end{itemize}
    \item Packaging:
    \begin{itemize}
        \item Sale Tasks
        \item Competitive Advantages
        \item Product Safety
    \end{itemize}
    \item Labeling:
    \begin{itemize}
        \item Identifies
        \item Describes
        \item Promotes
    \end{itemize}

    \item Product Support Service:
    \begin{enumerate}
        \item Survey customers to determine satisfaction with current services and any desired new services
        \item Assess costs of providing desired services
        \item Develop a package of services to delight customers and yield profits
    \end{enumerate}
    \item Product Line Decisions: 
    \item Product Mix Decisions:
    \item Characteristics of Services:
    \begin{itemize}
        \item Intangibility: Services cannot be seen, tasted, felt, heard, or smelled before purchase
        \item Inseparability: Services are produced and consumed simultaneously, making it difficult to separate the service provider from the service itself
        \item Variability: The quality of services can vary significantly depending on who provides them, when, and where
        \item Perishability: Services cannot be stored for later use; they are consumed at the time of production
    \end{itemize}
    \item Marketing Strategies for Service: Firms
    \begin{itemize}
        \item Managing Service Differentiation: Creating a unique service offering that stands out from competitors
        \item Managing Service Quality: Ensuring consistent and high-quality service delivery to meet customer expectations
        \item Managing Service Productivity: Balancing service quality with efficiency to maximize profitability
    \end{itemize}
\end{itemize}

\section{Product Portfolio Management}
\begin{itemize}
    \item Product Portfolio: The collection of all products and services offered by a company
\end{itemize}
\subsection{Product Life Cycle (PLC)}
\begin{itemize}
    \item Product Life Cycle (PLC): The stages a product goes through from introduction to decline
    \item Stages:
    \begin{itemize}
        \item Development (R\&D + NPD)
        \item Introduction/Launch
        \item Growth
        \item Maturity
        \item Saturation
        \item Decline/Withdrawal
    \end{itemize}
\end{itemize}
\subsubsection{Introduction Stage}
\begin{itemize}
    \item To do: Advertise and promote the product to create awareness and generate interest
    \item Monitoring: Initial sales and customer feedback to assess market acceptance
    \item Goal: Maximize publicity and build a customer base
    \item Cost: High cost, low sales
    \item Duration: Depend on type of product
\end{itemize}
\subsubsection{Growth Stage}
\begin{itemize}
    \item Performance: Consumer awareness increases, sales grow rapidly, revenue increases
    \item Cost: (fixed cost and variable cost) Profit may be made
    \item To do: Monitoring the market, competitors, and customer feedback to adapt marketing strategies
\end{itemize}
\subsubsection{Maturity Stage}
\begin{itemize}
    \item Performance: Sales peak, market share high, competition intensifies
    \item Cost: Decline the cost of supporting the product
    \item To do: Monitoring the market, considering changing strategies to maintain market share
\end{itemize}
\subsubsection{Saturation Stage}
\begin{itemize}
    \item Performance: Supply exceeds demand, sales plateau or decline
    \item To do: Develop new strategies like searching new markets, modifying the product, or finding new uses for it
\end{itemize}
\subsubsection{Decline Stage}
\begin{itemize}
    \item Performance: Product becomes obsolete or faces significant competition, sales decline sharply
    \item Due to: Technological advancements, changing consumer preferences, or increased competition
\end{itemize}

\subsection{The BCG Matrix}
\begin{itemize}
    \item BCG Matrix: A tool for analyzing a company's product portfolio based on market growth and market share (Boston Consulting Group Matrix)
    \item Classification:
    \begin{itemize}
        \item Stars: High market share, high market growth; require significant investment to maintain position
        \item Cash Cows: High market share, low market growth; generate steady cash flow with minimal investment
        \item Dogs: Low market share, low market growth; may not be worth investing in, often considered for divestment
        \item Problem Children (Question Marks): Low market share, high market growth; require careful analysis to determine whether to invest or divest
    \end{itemize}
    \item Implications:
    \begin{itemize}
        \item Stars: Invest to maintain leadership and capitalize on growth opportunities
        \item Cash Cows: Optimize profitability and use cash flow to support other products
        \item Dogs: Consider divesting or repositioning to minimize losses
        \item Problem Children: Analyze potential for growth and market share; decide whether to invest or divest
    \end{itemize}
\end{itemize}

% Topic 3
\chapter{Innovation Management}
\section{Innovation}
\begin{itemize}
    \item Innovation: The process of translating an idea or invention into a good or service that creates value or for which customers will pay
    \item Innovation characteristics:
    \begin{itemize}
        \item Essence: Process from idea generation to commercialization, the adoption of change
        \item Radical change in tranditional ways vs. incremental change
        \item Generate: New device or somrthing new to society
    \end{itemize}
\end{itemize}
\end{document}