\documentclass[a4paper,12pt,openany]{book}
\usepackage{graphicx}
\usepackage{amsmath, amssymb, amsthm}
\usepackage{amsfonts}
\usepackage{extarrows}
\usepackage{booktabs}
\usepackage{makecell}
\usepackage{geometry}
\usepackage{float}
\usepackage{multirow,multicol}
\geometry{a4paper,left=2cm,right=2cm}
\begin{document}
\chapter{Introduction}
\begin{itemize}
    \item Product: Something sold by an enterprise to its customers
    \item Product Development: The set of activities beginning with the perception of a market opportunity and ending in the production, sale, and delivery of a product
\end{itemize}
\section{Characteristics of Successful Product Development}
\begin{itemize}
    \item Dimensions to assess performance:
    \begin{itemize}
        \item Product Quality: The degree to which a product meets customer expectations
        \item Product Cost: The total cost incurred in producing and delivering the product
        \item Development Time: The time taken from the initial concept to the market launch
        \item Development Cost: The total cost incurred in the product development process
        \item Development Capability: The ability of the organization to develop products effectively and efficiently
    \end{itemize}
\end{itemize}

\section{Participants in Product Development}
\begin{itemize}
    \item Central participants:
    \begin{itemize}
        \item Marketing: Identifies market opportunities and customer needs
        \item Design: Creates the product concept and specifications
        \item Manufacturing: Plans and executes the production process
    \end{itemize}
    \item Project team: The collection of individuals developing a product
    \begin{itemize}
        \item Core team: A small group of individuals from different functions who work closely together throughout the project
        \item Extended team: Includes additional members from other functions who contribute at various stages of the project
    \end{itemize}
\end{itemize}

\section{Duration and Cost of Product Development}
\begin{itemize}
    \item Duration: 3 to 5 years for complex products
    \item Cost: $\propto$ people involved and time taken
\end{itemize}

\section{Challenges in Product Development}
\begin{itemize}
    \item Challanges characteristics:
    \begin{itemize}
        \item Trade-offs: Balancing quality, cost, time, and capability
        \item Dynamics: Adapting to changing market conditions and technologies
        \item Details: Managing the complexity of product specifications and requirements
        \item Time Pressure: Meeting tight deadlines while maintaining quality
        \item Economics: Ensuring the product is financially viable
    \end{itemize}
    \item Intrinsic attributes that make product development attractive:
    \begin{itemize}
        \item Creation
        \item Satisfaction of societal and individual needs
        \item Team diversity
        \item Team spirit
    \end{itemize}
\end{itemize}

\chapter{Product Development Process and Organization}
\section{Product Development Process}
\begin{itemize}
    \item Product development process: The sequence of steps or activities that an enterprise employs to conceive, design, and commercialize a product
    \item Advantages of having a well-defined process:
    \begin{itemize}
        \item Quality assurance: Ensures that the product meets customer expectations
        \item Coordination: Facilitates communication and collaboration among team members
        \item Planning: Helps in resource allocation and scheduling
        \item Management: Provides a framework for monitoring progress and making adjustments
        \item Improvement: Enables learning from past projects to enhance future performance
    \end{itemize}
    \item Six phases of the generic development process:
    \begin{enumerate}
        \item[0.] Planning
        \item Concept Development
        \item System-Level Design
        \item Detail Design
        \item Testing and Refinement
        \item Production Ramp-Up
    \end{enumerate}
\end{itemize}

\section{Concept Development: The Front-End Process}
\begin{itemize}
    \item Identifying customer needs
    \item Establishing target specifications
    \item Concept generation
    \item Concept selection
    \item Concept testing
    \item Setting final specifications
    \item Project planning
    \item Economic analysis
    \item Benchmarking of competitive products
    \item Modeling and prototyping
\end{itemize}

\chapter{Opportunity Identification}
\section{Opportunity}
\begin{itemize}
    \item Opportunity: An idea for a new product
\end{itemize}
\subsubsection{Types of Opportunities}
\begin{itemize}
    \item Two dimensions:
    \begin{itemize}
        \item Solution (Technology, Method, Process)
        \item Need (Market, Customer, User)
    \end{itemize}
\end{itemize}

\section{Tournament Structure of Opportunity Identification}
\begin{itemize}
    \item Goal: To take the opportunity articulated in the mission statement and do everything possible to assure it becomes the best product it can be
    \item 3 Basic Ways for Effective Opportunity Tournaments:
    \begin{itemize}
        \item Generate a large number of opportunities
        \item Seek high quality of the opportunities generated
        \item Create high variance in the quality of opportunities
    \end{itemize}
\end{itemize}

\section{Opportunity Identification Process}
\begin{itemize}
    \item 6 steps:
    \begin{enumerate}
        \item Establish a charter
        \item Generate and sense many opportunities
        \item Screen opportunities
        \item Develop promising opportunities
        \item Select exceptional opportunities
        \item Reflect on the results and the process
    \end{enumerate}
\end{itemize}
\subsubsection{Step 1: Establish a Charter}
\begin{itemize}
    \item Charter: Articulate the goals and establish the boundary conditions for an innovation effort
    \item Charter $\approx$ Mission statement for a new product
    \item Requires:
    \begin{itemize}
        \item Resolving a tension between leaving the innovation problem unconstrained
        \item Specifying a direction that is likely to meet the goals of the team and organization
    \end{itemize}
    \item Recommended:
    \begin{itemize}
        \item The innovation charter be broad. Benefit is that opportunities that may otherwise have never been considered will challenge the team’s assumptions about what kinds of opportunities it should pursue
    \end{itemize}
\end{itemize}
\subsubsection{Step 2: Generate and Sense Many Opportunities}
\begin{itemize}
    \item Opportunities from various sources:
    \begin{itemize}
        \item Internal sources: Employees, R\&D, existing products
        \item External sources: Customers, competitors, market trends, technology advancements
    \end{itemize}
    \item Techniques for Generating Opportunities:
    \begin{itemize}
        \item Follow a Personal Passion
        \item Compile Bug Lists
        \item Pull Opportunities from Capabilities (VRIN)
        \begin{itemize}
            \item Valuable
            \item Rare
            \item Inimitable
            \item Non-substitutable
        \end{itemize}
        \item Study Customers (Find latent needs)
        \item Consider Implications of Trends
        \item Imitate, but Better
        \begin{itemize}
            \item Media and marketing activities of other firms
            \item De-commoditize a commodity
            \item Drive an innovation ``down market''
            \item Import geographically isolated innovations
        \end{itemize}
        \item Mine Your Sources (Mainly external sources)
        \begin{itemize}
            \item Lead users
            \item Representation in social networks
            \item Universities and government laboratories
            \item Online idea submission
        \end{itemize}
    \end{itemize}
\end{itemize}

\subsubsection{Step 3: Screen Opportunities}
\begin{itemize}
    \item Goal: To eliminate opportunities that are highly unlikely to result in the creation of value and to focus attention on the opportunities worthy of further investigation
    \item \textbf{Not} to pick the single best opportunity
    \item Two methods for screening opportunities:
    \begin{itemize}
        \item Web-based surveys
        \begin{itemize}
            \item Fairness: Participants don't know the authors of the opportunities
            \item At least 6 independent judgements, Recommended 10
        \end{itemize}
        \item Workshops with ``multivoting''
        \begin{itemize}
            \item Each participant presents one or more opportunities
            \item Group multivotes on the opportunities
            \item About 50 opportunities are good for a workshop. Can use a web-based survey to screen down to 50
            \item Also pay attention to those with only a few very enthusiastic supporters
        \end{itemize}
    \end{itemize}
\end{itemize}

\subsubsection{Step 4: Develop Promising Opportunities}
\begin{itemize}
    \item Goal: To resolve the greatest uncertainty surrounding each one at the lowest cost in time and money
    \item Determine:
    \begin{itemize}
        \item The major uncertainties regarding the success of each opportunity
        \item The tasks you could take to resolve the uncertainties
        \item The approximate cost of each task
    \end{itemize}
    \item Invest modest levels of resources in developing a few of them
    \item Additional tasks (customer interviews, testing of existing products, etc.)
\end{itemize}

\subsubsection{Step 5: Select Exceptional Opportunities}
\begin{itemize}
    \item Method: RWW (Real, Win, Worth it)
    \begin{itemize}
        \item Real: Is there a real market and a real product?
        \item Win: Can we win? Can our product or service be competitive? Can we succeed as a company?
        \item Worth it: Is it worth doing? Is the return adequate and the risk acceptable?
    \end{itemize}
\end{itemize}

\subsubsection{Step 6: Reflect on the Results and the Process}
\begin{itemize}
    \item How many of the opportunities identified came from internal sources versus external sources?
    \item Did we consider dozens or hundreds of opportunities?
    \item Was the innovation charter too narrowly focused?
    \item Were our filtering criteria biased, or largely based on the best possible estimates of eventual product success?
    \item Are the resulting opportunities exciting to the team?
\end{itemize}

\chapter{Product Planning}
\begin{itemize}
    \item An activity that considers both the current product line and the potential portfolio of projects that an organization might pursue
\end{itemize}
\section{Product Planning Process}
\begin{itemize}
    \item Product plan: Identifies the portfolio of products to be developed by the organization and the timing of their introduction to the market
    \item Inefficiencies (no good product plan):
    \begin{itemize}
        \item Inadequate coverage of target markets with competitive products
        \item Poor timing of market introductions of products
        \item Mismatches between aggregate development capacity and the number of projects pursued
        \item Poor distribution of resources, with some projects overstaffed and others understaffed
        \item Initiation and subsequent cancellation of ill-conceived projects
        \item Frequent changes in the directions of projects
    \end{itemize}
\end{itemize}
\subsection{Types of Product Plans}
\begin{itemize}
    \item New Product Platforms: A set of products that share a common architecture and components, allowing for economies of scale and scope
    \item Derivatives of existing product platforms: Products that are based on existing platforms but have modifications or enhancements
    \item Incremental improvements to existing products: Small enhancements or modifications to existing products to improve performance, quality, or features
    \item Fundamentally new products: Products that are significantly different from existing offerings and may require new technologies or processes
\end{itemize}
\section{Process}
\begin{enumerate}
    \item Identify Opportunities
    \item Evaluate and Prioritize Projects
    \item Allocate Resources and Plan Timing
    \item Complete Pre-Project Planning
    \item Reflect on the Results and the Process
\end{enumerate}
% Page 73
\subsection{Step 1: Identify Opportunities}
\begin{itemize}
    \item Opportunity funnel
    \item Recommend: each promising opportunity be described in a short, coherent statement and that this information be collected in a database
\end{itemize}

\subsection{Step 2: Evaluate and Prioritize Projects}
\begin{itemize}
    \item To select the most promising projects to pursue
    \item Basic perspectives:
    \begin{itemize}
        \item Competitive strategy
        \item Market segmentation 
        \item Technological trajectories
        \item Project platform
    \end{itemize}
\end{itemize}
\subsubsection{Competitive Strategy}
\begin{itemize}
    \item Defines a basic approach to markets and products with respect to competitors
    \begin{itemize}
        \item Technology leadership
        \item Cost leadership
        \item Customer focus
        \item Imitate
    \end{itemize}
\end{itemize}
\subsubsection{Market Segmentation}
\begin{itemize}
    \item Allows the firm to consider the actions of competitors and the strength of the firm’s existing products with respect to each well-defined group of customers
    \item Can assess which product opportunities best address weaknesses in its own product line and which exploit weaknesses in the offerings of competitors
\end{itemize}
\subsubsection{Technological Trajectories}
\begin{itemize}
    \item Key: When to adopt a new basic technology in a product line
    \item Conceptual tool: Technology S-Curve
\end{itemize}
\subsubsection{Project Platform Planning}
\begin{itemize}
    \item The set of assets shared across a set of products
    \item Effective platform $\xrightarrow[Rapidly]{Easily}$ A variety of derivative products
    \item Key Strategy: Whether any project will develop a derivative product from an existing platform or develop an entirely new platform
    \item Technology roadmap: 
\end{itemize}

\subsubsection{Evaluating Fundamentally New Product Opportunities}
\begin{itemize}
    \item Market size (units/year$\times$average price)
    \item Market growth rate (percent per year)
    \item Competitive intensity (range of competitors and their strengths)
    \item Depth of the firm's existing knowledge of the market
    \item Depth of the firm's existing knowledge of the technology
    \item Fit with the firm's other products
    \item Fit with the firm's core assets and capabilities
    \item Potential for patents, trade secrets, or other barriers to competition
    \item Existence of a product champion within the firm
\end{itemize}

\subsubsection{Balancing the Portfolio}
\begin{itemize}
    \item Two specific dimensions:
    \begin{itemize}
        \item The extent to which the project involves a change in the product line
        \item The extent to which the project involves a change in production processes
    \end{itemize}
    \item Advantage:
    \begin{itemize}
        \item Be useful to illuminate imbalances in the portfolio of projects under consideration 
        \item In assessing the consistency between a portfolio of projects and the competitive strategy
    \end{itemize}
\end{itemize}

\subsection{Step 3: Allocate Resources and Plan Timing}
\subsubsection{Allocate Resources}
\begin{itemize}
    \item Aggregate Planning: Helps an organization make efficient use of its resources by pursuing only those projects that can reasonably be completed with the allocated resources
    \item Primary resource to be managed: The effort of the development staff (person-hours or person-months)
    \item Capacity utilization: 80\% to 90\%
\end{itemize}
\subsubsection{Project Timing}
\begin{itemize}
    \item Timing of product introductions
    \item Technology readiness
    \item Market readiness
    \item Competition
\end{itemize}
\subsubsection{The Product Plan}
\begin{itemize}
    \item The set of projects approved by the planning process, sequenced in time
    \item May include:
    \begin{itemize}
        \item A mix of fundamentally new products
        \item Platform projects
        \item Derivative projects of varying size
    \end{itemize}
\end{itemize}
\subsection{Step 4: Complete Pre-Project Planning}
\begin{itemize}
    \item By: Core team
    \item Product vision statement: A brief description of the product and its intended market
\end{itemize}
\subsubsection{Mission Statement}
\begin{itemize}
    \item Brief (one sentence) description of the product
    \item Benefit proposition
    \item Key business goals
    \item Target market(s) for the product
    \item Assumptions and constraints that guide the development effort
    \item Stakeholders
\end{itemize}
\subsubsection{Assumptions and Constraints}
\begin{itemize}
    \item Considers the strategies of several functional areas within the firm
    \item Consider:
    \begin{itemize}
        \item Manufacturing
        \item Service
        \item Environment
    \end{itemize}
\end{itemize}

\subsection{Step 5: Reflect on the Results and the Process}
\begin{itemize}
    \item Is the opportunity funnel collecting an exciting and diverse set of product opportunities?
    \item Does the product plan support the competitive strategy of the firm?
    \item Does the product plan address the most important current opportunities facing the firm?
    \item Are the total resources allocated to product development sufficient to pursue the firm's competitive strategy?
    \item Have creative ways of leveraging finite resources been considered, such as the use of product platforms, joint ventures, and partnerships with suppliers?
    \item $\cdots$
\end{itemize}

% Page 88/94
\chapter{Identifying Customer Needs}
\begin{itemize}
    \item Concept development:
    \begin{itemize}
        \item \textbf{Identify Customer Needs}
        \item Establish Target Specifications
        \item Generate Product Concepts
        \item Select Product Concepts
        \item Test Product Concepts
        \item Set Final Specifications
        \item Plan Downstream Development
    \end{itemize}
    \item Goal:
    \begin{itemize}
        \item Ensure that the product is focused on customer needs
        \item Identify latent or hidden needs as well as explicit needs
        \item Provide a fact base for justifying the product specifications
        \item Create an archival record of the needs activity of the development process
        \item Ensure that no critical customer need is missed or forgotten
        \item Develop a common understanding of customer needs among members of the development team
    \end{itemize}
    \item Philosophy: To create a high-quality information channel that runs directly between customers in the target market and the developers of the product    
\end{itemize}
\section{The Importance of Latent Needs}
\begin{itemize}
    \item Latent needs: Not yet widely recognized by most customers and not yet addressed by existing products
\end{itemize}
\section{The Process of Identifying Customer Needs}
\begin{itemize}
    \item 5 steps:
    \begin{enumerate}
        \item Gather raw data from customers
        \item Interpret the raw data in terms of customer needs
        \item Organize the needs into a hierarchy of primary, secondary, and tertiary needs
        \item Establish the relative importance of the needs
        \item Reflect on the results and the process
    \end{enumerate}
\end{itemize}
\subsection{Step 1: Gather raw data from customers}
\begin{itemize}
    \item Interview
    \item Focus group
    \item Observing the product in use
\end{itemize}
\subsubsection{Choosing Customers to Interview}
\begin{itemize}
    \item 10 to 50 times
    \item Lead user
    \item Extreme user
\end{itemize}
\subsubsection{The Art of Eliciting Customer Needs Data}
\begin{itemize}
    \item Effective interaction
    \begin{itemize}
        \item Go with the flow
        \item Use visual stimuli and props
        \item Suppress preconceived hypotheses about the product technology
        \item Have the customer demonstrate the product and/or typical tasks related to the product
        \item Be alert for surprises and the expression of latent needs
        \item Watch for nonverbal information
        \item Data privacy
    \end{itemize}
\end{itemize}
\subsubsection{Documenting Interactions with Customers}
\begin{itemize}
    \item Audio recording
    \item Notes
    \item Video recording
    \item Still photography
\end{itemize}

\subsection{Step 2: Interpret the raw data in terms of customer needs}
\begin{itemize}
    \item Express the need in terms of what the product has to do, not in terms of how it might do it
    \item Express the need as specifically as the raw data
    \item Use positive, not negative, phrasing
    \item Express the need as an attribute of the product
    \item Avoid the words \textbf{must} and \textbf{should}
\end{itemize}

\subsection{Step 3: Organize the needs into a hierarchy}
\begin{enumerate}
    \item Print or write each needs statement on a separate card or self-stick note
    \item Eliminate redundant statements
    \item Group the cards according to the similarity of the needs they express
    \item For each group, choose a label
    \item Consider creating supergroups consisting of two to five groups
    \item Review and edit the organized needs statements
\end{enumerate}

\subsection{Step 4: Establish the relative importance of the needs}
\begin{itemize}
    \item Relying on the consensus of the team members based on their experience with customers
    \item Basing the importance assessment on further customer surveys
\end{itemize}

\subsection{Step 5: Reflect on the results and the process}
\begin{itemize}
    \item Have we interacted with all of the important types of customers in our target market?
    \item Are we able to see beyond needs related only to existing products to capture the latent needs of our target customers?
    \item Are there areas of inquiry we should pursue in follow-up interviews or surveys?
    \item Which of the customers we spoke to would be good participants in our ongoing development efforts?
    \item What do we know now that we didn’t know when we started? Are we surprised by any of the needs?
    \item Did we involve everyone within our own organization who needs to deeply understand customer needs?
    \item How might we improve the process in future efforts?
\end{itemize}

\chapter{Product Specifications}
\section{Specifications}
\begin{itemize}
    \item Customer needs $\to$ ``Language of the customer''
    \item Specifications $\to$ ``Language of the engineer''
    \item Specifications: an unambiguous agreement on what the team will attempt to achieve to satisfy the customer needs
    \item Include: Metric, Value
\end{itemize}
\section{When Are Specifications Established}
\begin{itemize}
    \item Target specifications: Immediately after Identifying Customer Needs
    \item Final specifications: After concept is determined (tested)
\end{itemize}

\section{Establishing Target Specifications}
\begin{itemize}
    \item Target specifications: The goals of the development team, describing a product that the team believes would succeed in the marketplace
    \item 4 steps:
    \begin{itemize}
        \item Prepare the list of metrics
        \item Collect competitive benchmarking information
        \item Set ideal and marginally acceptable target values
        \item Reflect on the results and the process
    \end{itemize}
\end{itemize}
\subsubsection{Step 1: Prepare the list of metrics}
\begin{itemize}
    \item Metrics: Reflect as directly as possible the degree to which the product satisfies the customer needs
    \item Needs-metrics matrix: Represents the relationship between needs and metrics
    \item QFD (Quality Function Deployment): A tool to help ensure that the metrics are directly related to customer needs
    \item House of Quality: A matrix that relates customer needs to product specifications, helping to prioritize the metrics based on their importance to the customer
    \item Considered guidelines:
    \begin{itemize}
        \item Metrics should be complete
        \item Metrics should be dependent, not independent, variables
        \item Metrics should be practical
        \item Some needs cannot easily be translated into quantifiable metrics
        \item The metrics should include the popular criteria for comparison in the marketplace
    \end{itemize}
\end{itemize}
\subsubsection{Step 2: Collect competitive benchmarking information}
\begin{itemize}
    \item Competitive benchmarking chart: A table that lists the metrics and their values for competitive products, allowing the team to understand the current market standards
    \item 2 kind:
    \begin{itemize}
        \item Row: the metrics, Column: the competitive products (basic)
        \item Row: the customer needs, Column: the competitive products (advanced)
    \end{itemize}
\end{itemize}
\subsubsection{Step 3: Set ideal and marginally acceptable target values}
\begin{itemize}
    \item 2 Target values:
    \begin{itemize}
        \item Ideal value: The best result the team could hope for
        \item Marginally acceptable value: The value of the metric that would just barely make the product commercially viable
    \end{itemize}
    \item 5 ways to express:
    \begin{itemize}
        \item At least X
        \item At most X
        \item Between X and Y
        \item Exactly X
        \item A set of discrete values
    \end{itemize}
\end{itemize}
\subsubsection{Step 4: Reflect on the results and the process}
\begin{itemize}
    \item Are members of the team ``gaming''?
    \item Should the team consider offering multiple products or at least multiple options for the product to best match the particular needs of more than one market segment, or will one ``average'' product suffice?
    \item Are any specifications missing? Do the specifications reflect the characteristics that will dictate commercial success?
\end{itemize}

\section{Setting Final Specifications}
\begin{itemize}
    \item Final specifications: Revised target specifications after product concept selection
    \item 5 steps:
    \begin{itemize}
        \item Develop technical models of the product
        \item Develop a cost model of the product
        \item Refine the specifications, making trade-offs where necessary
        \item Flow down the specifications as appropriate
        \item Reflect on the results and the process
    \end{itemize}
\end{itemize}

% Page 131/138
\chapter{Concept Generation}
\section{The Activity of Concept Generation}
\begin{itemize}
    \item Concept generation: An approximate description of the technology, working principles, and form of the product
    \item Form (with a brief textual description):
    \begin{itemize}
        \item Sketch
        \item rough three-dimensional model
    \end{itemize}
\end{itemize}
\section{Structured Approaches}
\begin{itemize}
    \item 5 steps:
    \begin{enumerate}
        \item Clarify the problem
        \item Search externally
        \item Search internally
        \item Explore systematically
        \item Reflect on the results and the process
    \end{enumerate}
\end{itemize}

\chapter{Concept Selection}
\section{Concept Selection}
\begin{itemize}
    \item Concept selection: The process of evaluating concepts with respect to customer needs and other criteria, comparing the relative strengths and weaknesses of the concepts, and selecting one or more concepts for further investigation, testing, or development
\end{itemize}
\section{Methods of Concept Selection}
\begin{itemize}
    \item External decision
    \item Product champion
    \item Intuition
    \item Multivoting
    \item Online survey/crowdsourcing
    \item Pros and cons
    \item Prototype and test
    \item Decision matrices
\end{itemize}
\section{Advantages}
\begin{itemize}
    \item A customer-focused product
    \item A competitive design
    \item Better product-process coordination
    \item Reduced time to product introduction
    \item Effective group decision making
    \item Documentation of the decision process
\end{itemize}
\section{Overview}
\begin{itemize}
    \item 2 phases:
    \begin{itemize}
        \item Concept screening: A quick, rough evaluation of the concepts to eliminate those that are clearly unacceptable
        \item Concept scoring: A more detailed evaluation of the remaining concepts to select the best one(s)
    \end{itemize}
    \item 6 steps for each phase:
    \begin{enumerate}
        \item Prepare the selection matrix
        \item Rate the concepts
        \item Rank the concepts
        \item Combine and improve the concepts
        \item Select one or more concepts
        \item Reflect on the results and the process
    \end{enumerate}
\end{itemize}

\chapter{Concept Testing}
\begin{itemize}
    \item 7 steps:
    \begin{enumerate}
        \item Define the purpose of the concept test
        \item Choose a survey population
        \item Select a survey format
        \item Communicate the concept
        \item Measure customer response
        \item Interpret the results
        \item Reflect on the results and the process
    \end{enumerate}
\end{itemize}

\end{document}