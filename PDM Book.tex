\documentclass[a4paper,12pt,openany]{book}
\usepackage{graphicx}
\usepackage{amsmath, amssymb, amsthm}
\usepackage{amsfonts}
\usepackage{extarrows}
\usepackage{booktabs}
\usepackage{makecell}
\usepackage{geometry}
\usepackage{float}
\usepackage{multirow,multicol}
\geometry{a4paper,left=2cm,right=2cm}
\begin{document}
\chapter{Introduction}
\begin{itemize}
    \item Product: Something sold by an enterprise to its customers
    \item Product Development: The set of activities beginning with the perception of a market opportunity and ending in the production, sale, and delivery of a product
\end{itemize}
\section{Characteristics of Successful Product Development}
\begin{itemize}
    \item Dimensions to assess performance:
    \begin{itemize}
        \item Product Quality: The degree to which a product meets customer expectations
        \item Product Cost: The total cost incurred in producing and delivering the product
        \item Development Time: The time taken from the initial concept to the market launch
        \item Development Cost: The total cost incurred in the product development process
        \item Development Capability: The ability of the organization to develop products effectively and efficiently
    \end{itemize}
\end{itemize}

\section{Participants in Product Development}
\begin{itemize}
    \item Central participants:
    \begin{itemize}
        \item Marketing: Identifies market opportunities and customer needs
        \item Design: Creates the product concept and specifications
        \item Manufacturing: Plans and executes the production process
    \end{itemize}
    \item Project team: The collection of individuals developing a product
    \begin{itemize}
        \item Core team: A small group of individuals from different functions who work closely together throughout the project
        \item Extended team: Includes additional members from other functions who contribute at various stages of the project
    \end{itemize}
\end{itemize}

\section{Duration and Cost of Product Development}
\begin{itemize}
    \item Duration: 3 to 5 years for complex products
    \item Cost: $\propto$ people involved and time taken
\end{itemize}

\section{Challenges in Product Development}
\begin{itemize}
    \item Challanges characteristics:
    \begin{itemize}
        \item Trade-offs: Balancing quality, cost, time, and capability
        \item Dynamics: Adapting to changing market conditions and technologies
        \item Details: Managing the complexity of product specifications and requirements
        \item Time Pressure: Meeting tight deadlines while maintaining quality
        \item Economics: Ensuring the product is financially viable
    \end{itemize}
    \item Intrinsic attributes that make product development attractive:
    \begin{itemize}
        \item Creation
        \item Satisfaction of societal and individual needs
        \item Team diversity
        \item Team spirit
    \end{itemize}
\end{itemize}

\chapter{Product Development Process and Organization}
\section{Product Development Process}
\begin{itemize}
    \item Product development process: The sequence of steps or activities that an enterprise employs to conceive, design, and commercialize a product
    \item Advantages of having a well-defined process:
    \begin{itemize}
        \item Quality assurance: Ensures that the product meets customer expectations
        \item Coordination: Facilitates communication and collaboration among team members
        \item Planning: Helps in resource allocation and scheduling
        \item Management: Provides a framework for monitoring progress and making adjustments
        \item Improvement: Enables learning from past projects to enhance future performance
    \end{itemize}
    \item Six phases of the generic development process:
    \begin{enumerate}
        \item[0.] Planning
        \item Concept Development
        \item System-Level Design
        \item Detail Design
        \item Testing and Refinement
        \item Production Ramp-Up
    \end{enumerate}
\end{itemize}

\section{Concept Development: The Front-End Process}
\begin{itemize}
    \item Identifying customer needs
    \item Establishing target specifications
    \item Concept generation
    \item Concept selection
    \item Concept testing
    \item Setting final specifications
    \item Project planning
    \item Economic analysis
    \item Benchmarking of competitive products
    \item Modeling and prototyping
\end{itemize}

\chapter{Opportunity Identification}
\section{Opportunity}
\begin{itemize}
    \item Opportunity: An idea for a new product
\end{itemize}
\subsubsection{Types of Opportunities}
\begin{itemize}
    \item Two dimensions:
    \begin{itemize}
        \item Solution (Technology, Method, Process)
        \item Need (Market, Customer, User)
    \end{itemize}
\end{itemize}
\end{document}