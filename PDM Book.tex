\documentclass[a4paper,12pt,openany]{book}
\usepackage{graphicx}
\usepackage{amsmath, amssymb, amsthm}
\usepackage{amsfonts}
\usepackage{extarrows}
\usepackage{booktabs}
\usepackage{makecell}
\usepackage{geometry}
\usepackage{float}
\usepackage{multirow,multicol}
\geometry{a4paper,left=2cm,right=2cm}
\begin{document}
\chapter{Introduction}
\begin{itemize}
    \item Product: Something sold by an enterprise to its customers
    \item Product Development: The set of activities beginning with the perception of a market opportunity and ending in the production, sale, and delivery of a product
\end{itemize}
\section{Characteristics of Successful Product Development}
\begin{itemize}
    \item Dimensions to assess performance:
    \begin{itemize}
        \item Product Quality: The degree to which a product meets customer expectations
        \item Product Cost: The total cost incurred in producing and delivering the product
        \item Development Time: The time taken from the initial concept to the market launch
        \item Development Cost: The total cost incurred in the product development process
        \item Development Capability: The ability of the organization to develop products effectively and efficiently
    \end{itemize}
\end{itemize}

\section{Participants in Product Development}
\begin{itemize}
    \item Central participants:
    \begin{itemize}
        \item Marketing: Identifies market opportunities and customer needs
        \item Design: Creates the product concept and specifications
        \item Manufacturing: Plans and executes the production process
    \end{itemize}
    \item Project team: The collection of individuals developing a product
    \begin{itemize}
        \item Core team: A small group of individuals from different functions who work closely together throughout the project
        \item Extended team: Includes additional members from other functions who contribute at various stages of the project
    \end{itemize}
\end{itemize}

\section{Duration and Cost of Product Development}
\begin{itemize}
    \item Duration: 3 to 5 years for complex products
    \item Cost: $\propto$ people involved and time taken
\end{itemize}

\section{Challenges in Product Development}
\begin{itemize}
    \item Challanges characteristics:
    \begin{itemize}
        \item Trade-offs: Balancing quality, cost, time, and capability
        \item Dynamics: Adapting to changing market conditions and technologies
        \item Details: Managing the complexity of product specifications and requirements
        \item Time Pressure: Meeting tight deadlines while maintaining quality
        \item Economics: Ensuring the product is financially viable
    \end{itemize}
    \item Intrinsic attributes that make product development attractive:
    \begin{itemize}
        \item Creation
        \item Satisfaction of societal and individual needs
        \item Team diversity
        \item Team spirit
    \end{itemize}
\end{itemize}

\chapter{Product Development Process and Organization}
\section{Product Development Process}
\begin{itemize}
    \item Product development process: The sequence of steps or activities that an enterprise employs to conceive, design, and commercialize a product
    \item Advantages of having a well-defined process:
    \begin{itemize}
        \item Quality assurance: Ensures that the product meets customer expectations
        \item Coordination: Facilitates communication and collaboration among team members
        \item Planning: Helps in resource allocation and scheduling
        \item Management: Provides a framework for monitoring progress and making adjustments
        \item Improvement: Enables learning from past projects to enhance future performance
    \end{itemize}
    \item Six phases of the generic development process:
    \begin{enumerate}
        \item[0.] Planning
        \item Concept Development
        \item System-Level Design
        \item Detail Design
        \item Testing and Refinement
        \item Production Ramp-Up
    \end{enumerate}
\end{itemize}

\section{Concept Development: The Front-End Process}
\begin{itemize}
    \item Identifying customer needs
    \item Establishing target specifications
    \item Concept generation
    \item Concept selection
    \item Concept testing
    \item Setting final specifications
    \item Project planning
    \item Economic analysis
    \item Benchmarking of competitive products
    \item Modeling and prototyping
\end{itemize}

\chapter{Opportunity Identification}
\section{Opportunity}
\begin{itemize}
    \item Opportunity: An idea for a new product
\end{itemize}
\subsubsection{Types of Opportunities}
\begin{itemize}
    \item Two dimensions:
    \begin{itemize}
        \item Solution (Technology, Method, Process)
        \item Need (Market, Customer, User)
    \end{itemize}
\end{itemize}

\section{Tournament Structure of Opportunity Identification}
\begin{itemize}
    \item Goal: To take the opportunity articulated in the mission statement and do everything possible to assure it becomes the best product it can be
    \item 3 Basic Ways for Effective Opportunity Tournaments:
    \begin{itemize}
        \item Generate a large number of opportunities
        \item Seek high quality of the opportunities generated
        \item Create high variance in the quality of opportunities
    \end{itemize}
\end{itemize}

\section{Opportunity Identification Process}
\begin{itemize}
    \item 6 steps:
    \begin{enumerate}
        \item Establish a charter
        \item Generate and sense many opportunities
        \item Screen opportunities
        \item Develop promising opportunities
        \item Select exceptional opportunities
        \item Reflect on the results and the process
    \end{enumerate}
\end{itemize}
\subsubsection{Step 1: Establish a Charter}
\begin{itemize}
    \item Charter: Articulate the goals and establish the boundary conditions for an innovation effort
    \item Charter $\approx$ Mission statement for a new product
    \item Requires:
    \begin{itemize}
        \item Resolving a tension between leaving the innovation problem unconstrained
        \item Specifying a direction that is likely to meet the goals of the team and organization
    \end{itemize}
    \item Recommended:
    \begin{itemize}
        \item The innovation charter be broad. Benefit is that opportunities that may otherwise have never been considered will challenge the team’s assumptions about what kinds of opportunities it should pursue
    \end{itemize}
\end{itemize}
\subsubsection{Step 2: Generate and Sense Many Opportunities}
\begin{itemize}
    \item Opportunities from various sources:
    \begin{itemize}
        \item Internal sources: Employees, R\&D, existing products
        \item External sources: Customers, competitors, market trends, technology advancements
    \end{itemize}
    \item Techniques for Generating Opportunities:
    \begin{itemize}
        \item Follow a Personal Passion
        \item Compile Bug Lists
        \item Pull Opportunities from Capabilities (VRIN)
        \begin{itemize}
            \item Valuable
            \item Rare
            \item Inimitable
            \item Non-substitutable
        \end{itemize}
        \item Study Customers (Find latent needs)
        \item Consider Implications of Trends
        \item Imitate, but Better
        \begin{itemize}
            \item Media and marketing activities of other firms
            \item De-commoditize a commodity
            \item Drive an innovation ``down market''
            \item Import geographically isolated innovations
        \end{itemize}
        \item Mine Your Sources (Mainly external sources)
        \begin{itemize}
            \item Lead users
            \item Representation in social networks
            \item Universities and government laboratories
            \item Online idea submission
        \end{itemize}
    \end{itemize}
\end{itemize}

\subsubsection{Step 3: Screen Opportunities}
\begin{itemize}
    \item Goal: To eliminate opportunities that are highly unlikely to result in the creation of value and to focus attention on the opportunities worthy of further investigation
    \item \textbf{Not} to pick the single best opportunity
    \item Two methods for screening opportunities:
    \begin{itemize}
        \item Web-based surveys
        \begin{itemize}
            \item Fairness: Participants don't know the authors of the opportunities
            \item At least 6 independent judgements, Recommended 10
        \end{itemize}
        \item Workshops with ``multivoting''
        \begin{itemize}
            \item Each participant presents one or more opportunities
            \item Group multivotes on the opportunities
            \item About 50 opportunities are good for a workshop. Can use a web-based survey to screen down to 50
            \item Also pay attention to those with only a few very enthusiastic supporters
        \end{itemize}
    \end{itemize}
\end{itemize}

\subsubsection{Step 4: Develop Promising Opportunities}
\begin{itemize}
    \item Goal: To resolve the greatest uncertainty surrounding each one at the lowest cost in time and money
    \item Determine:
    \begin{itemize}
        \item The major uncertainties regarding the success of each opportunity
        \item The tasks you could take to resolve the uncertainties
        \item The approximate cost of each task
    \end{itemize}
    \item Invest modest levels of resources in developing a few of them
    \item Additional tasks (customer interviews, testing of existing products, etc.)
\end{itemize}

\subsubsection{Step 5: Select Exceptional Opportunities}
\begin{itemize}
    \item Method: RWW (Real, Win, Worth it)
    \begin{itemize}
        \item Real: Is there a real market and a real product?
        \item Win: Can we win? Can our product or service be competitive? Can we succeed as a company?
        \item Worth it: Is it worth doing? Is the return adequate and the risk acceptable?
    \end{itemize}
\end{itemize}

\subsubsection{Step 6: Reflect on the Results and the Process}
\begin{itemize}
    \item How many of the opportunities identified came from internal sources versus external sources?
    \item Did we consider dozens or hundreds of opportunities?
    \item Was the innovation charter too narrowly focused?
    \item Were our filtering criteria biased, or largely based on the best possible estimates of eventual product success?
    \item Are the resulting opportunities exciting to the team?
\end{itemize}

\chapter{Product Planning}
\begin{itemize}
    \item An activity that considers both the current product line and the potential portfolio of projects that an organization might pursue
\end{itemize}
\section{Product Planning Process}
\begin{itemize}
    \item Product plan: Identifies the portfolio of products to be developed by the organization and the timing of their introduction to the market
    \item Inefficiencies (no good product plan):
    \begin{itemize}
        \item Inadequate coverage of target markets with competitive products
        \item Poor timing of market introductions of products
        \item Mismatches between aggregate development capacity and the number of projects pursued
        \item Poor distribution of resources, with some projects overstaffed and others understaffed
        \item Initiation and subsequent cancellation of ill-conceived projects
        \item Frequent changes in the directions of projects
    \end{itemize}
\end{itemize}
\subsection{Types of Product Plans}
\begin{itemize}
    \item New Product Platforms: A set of products that share a common architecture and components, allowing for economies of scale and scope
    \item Derivatives of existing product platforms: Products that are based on existing platforms but have modifications or enhancements
    \item Incremental improvements to existing products: Small enhancements or modifications to existing products to improve performance, quality, or features
    \item Fundamentally new products: Products that are significantly different from existing offerings and may require new technologies or processes
\end{itemize}
\subsection{Process}
\begin{enumerate}
    \item Identify Opportunities
    \item Evaluate and Prioritize Projects
    \item Allocate Resources and Plan Timing
    \item Complete Pre-Project Planning
    \item Product Development Process
\end{enumerate}
% Page 73
\subsubsection{Step 1: Identify Opportunities}
\end{document}